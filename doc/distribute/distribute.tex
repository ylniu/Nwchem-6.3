% $Id: distribute.tex 19708 2010-10-29 18:04:21Z d3y133 $

\documentstyle[fullpage,12pt,fleqn]{article}
\setlength{\parskip}{6pt}

\begin{document}

\begin{center}
{\bf\Large NWCHEM Distribution Information}
\end{center}

\section{Automatic Builds at PNNL}
Currently the automatic builds are run via cron jobs and work for the
most part unless someone has checked in code that really doesn't
work. The scripts simply build the code and do not do any regression on
the binary they produce.  These scripts always keep older versions of
the NWChem binaries.  We have started the development of some QA
procedures and I will incorporate a subset of this QA to into the
automatic builds so that we can have some small assurance that the new
binary will work as planned.  Automatic builds are produced on a
revolving schedule on various machines at PNNL.  The schedule
information will eventually be a web page on the internal EMSL site
but for now it is kept in a file located at:\\
/msrc/proj/nwchem/build\_readme

\subsection{AFS aware systems}

The AFS @sys technology is used to distinguish between operating
systems and the specific tags are used to distinguish between systems
with the same operating system and different characteristics.  For
example, IRIX 6.x runs on indigo's, origins, etc.  nwchem.R4.4k and
nwchem.R10k exist in the same IRIX 6.x @sys directory.

These machines have binaries located in three directories:
\begin{itemize}
\item The default release binary for general use is in :{\tt
/msrc/proj/nwchem/bin}.  This directory is updated by hand when deemed
necessary.
\item The release tree build (usually weekly) is in:{\tt /msrc/proj/nwchem/binr}
\item The development tree build (usually twice a week) is in:{\tt /msrc/proj/nwchem/bind}
\end{itemize}

I will eventually move this to DFS but I cannot get DFS credentials in
a cron job yet.  

\subsection{On NWMPP1}
On NWMPP1 there is a user account ``nwchem'' where the automatic
builds 

\section{Off-site}
\subsection{Current Procedure}
Rick does it with each site.
\subsection{Future}
I want to put in place the technology that requires the user to fill
out a form and then he will get the user agreement in the US mail.
Once he returns the user agreement we send him to a page where he will
bring up his registration information and enter a password of sorts
and get to a page where he can down-load the binaries that are
available.

A similar mechanism could be used to distribute the tar file of the
current release version of NWChem.

\end{document}

